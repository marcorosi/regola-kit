% modello documentale di nicola santi ego@nicolasanti.it

\documentclass[pdftex,a4paper,11pt,twoside,openright]{book}

\usepackage[italian]{babel}
\usepackage[T1]{fontenc}
\usepackage[utf8]{inputenc}
\usepackage{indentfirst}

% decora i titoli dei capitoli
\usepackage[Lenny]{fncychap}

% inserimento di immagini come effetto indesiderato
% produce solo pdf come output e non più dvi.
\usepackage[pdftex]{graphicx}
\DeclareGraphicsExtensions{.pdf,.png,.jpg}
\graphicspath{{./images/}}

\newcommand{\HRule}{\rule{\linewidth}{0.5mm}}

% floatings, caption laterali
\usepackage{sidecap}

% immagini avvolte all'interno del testo
\usepackage{wrapfig}

% citazioni in stile Hardvare
\usepackage{natbib}
\bibpunct{(}{)}{;}{a}{,}{,}

% forza LaTeX ad una spaziatura fra parole non inglese
\frenchspacing 

% per inserire del codice sorgente
\usepackage{listings}
\usepackage{color}
\definecolor{light-gray}{gray}{0.95}


\lstnewenvironment{java}[1][]
{\lstset{language=Java,numbers=left, numberstyle=\tiny, stepnumber=2, numbersep=5pt,backgroundcolor=\color{light-gray},breaklines=true,breakatwhitespace=true,breakindent=0pt,basicstyle=\scriptsize,#1}}
{}

\lstnewenvironment{bash}[1][]
{\lstset{language=Bash,numbers=left, numberstyle=\tiny, stepnumber=2, numbersep=5pt,backgroundcolor=\color{light-gray},breaklines=true,breakatwhitespace=true,breakindent=0pt,basicstyle=\scriptsize,#1}}
{}

\lstnewenvironment{xml}[1][]
{\lstset{language=Xml,numbers=left, numberstyle=\tiny, stepnumber=2, numbersep=5pt,backgroundcolor=\color{light-gray},breaklines=true,breakatwhitespace=true,breakindent=0pt,basicstyle=\scriptsize,#1}}
{}



% indentazione del primo paragrafo
\setlength{\parindent}{18pt}

% serve per gestire le tabelle
\usepackage{array}
\usepackage{multirow}
\usepackage{booktabs}

% per inserire l'euro
\usepackage[official]{eurosym}

% liste più compatte
\usepackage{mdwlist}

% liste inline
\usepackage{paralist}

% miglioare i riferimenti
\usepackage[italian]{varioref}
%\def\reftextfaraway#1{a pagina~\pageref{#1}}%
%\def\reftextpagerange#1#2{nelle pagine~\pageref{#1}--\pageref{#2}}%

% hyper link in LaTex
\usepackage{html}

% indici
\usepackage{makeidx}
\makeindex

% un simpatico box

\usepackage{dashbox}

\newenvironment{nota}
{\rule{1ex}{1ex}\hspace{\stretch{1}}}
{\hspace{\stretch{1}}\rule{1ex}{1ex}}



\begin{document} 

%-----------TOP MATTER

%\title{Pojo's in action} 

%\author{Nicola Santi\\
%  Università di Bologna\\
%  \texttt{ego@nicolasanti.it}}
%\date{\today}

%\maketitle 
\begin{titlepage}
 
\begin{center}
 
 
% Upper part of the page
%\includegraphics[width=0.40\textwidth]{regola-kit}\\[1cm]
 
\textsc{\LARGE }\\[1.5cm]
 
\textsc{\Large guida per lo sviluppatore }\\[0.5cm]
 
 % Title
\HRule \\[0.4cm]
{ \huge \bfseries Regola kit}\\[0.4cm]
 
\HRule \\[1.5cm]

 
% Author and supervisor
\begin{minipage}{0.9\textwidth}
\begin{flushright} \large
\emph{Autori:}\\
Nicola \textsc{Santi}
\end{flushright}
\end{minipage}
 
\vfill
 
% Bottom of the page
{\large \today}
 
\end{center}
 
\end{titlepage}

% -----------TOC
\tableofcontents
%\listoffigures
%\listoftables



\chapter*{Prefazione}

Il panorama delle tecnologie Java si presenta allo sviluppatore rigoglioso e selvaggio come doveva apparire agli occhi dei primi europei la foresta amazzonica; centinaia di librerie si contendono la stessa funzionalità ed assolvono il loro compito in modo spesso diverso e non di rado incompatibile. Una menzione particolare merita lo sviluppo di applicazioni web per cui esistono paradigmi e librerie a bizzeffe in una sfilata allegra e caotica di sigle (Servlet, JSP, Struts, JSF, Ajax, Tapestry e molte molte altre) che lasciano spesso interdetto il neofita, che di fronte a tanta abbondanza si ritrova spesso a scegliere a caso lasciandosi guidare dall'ultimo articolo sfogliato. Purtroppo la statistica non perdona e spesso la scelta si rivela al di sotto delle aspettative ed è qui, sopra i tavoli di una qualche riunione, che qualcuno cede allo scoramento e pronuncia la fatidica frase: <<Riscriviamolo noi>>. Ahi quanti si sono imbarcati nell'impresa funesta di realizzare il proprio framework in casa e quanti ancora ne stanno pagando un caro prezzo in termini di produttività, innovazione e stabilità.

Regola kit è un tentativo di risposta alla domanda “Quale tecnologie conviene utilizzare per un'applicazione web con un'alta interazione utente e mole di dati ampissima?”. Ad esempio per un gestionale web, il dominio che ha fatto da culla per il progetto. Conviene subito fare chiarezza su un punto: Regola kit non è un framework. Il suo contenuto è essenzialmente esperienza. Il filosofo Hume suggeriva un metodo curioso per selezionare i libri in un biblioteca; scartare tutti quelli che contengono speculazioni astratte, trattati di logica e quant'altro per limitarsi a considerare solo quei testi che riportano esperienze reali degli autori. Questo criterio di selezione vale più che mai in un mondo in cui librerie e progetti (open source o commerciali) sorgono e tramontano nello spazio di un mattino e fornisce la chiave per selezionare ciò che in potenza appare perfetto da quello che realmente è in grado di consentire lo sviluppo con un livello di produttività e gratificazione decenti. Ad esempio per la sola persistenza esistono decine di framework ma quello scelto per Regola kit (Hibernate) è l'unico che ha superato la prova su database piuttosto estesi, il giusto compromesso tra flessibilità ed automatismo. Naturalmente per altri contesti Hibernate potrebbe non essere la soluzione migliore e per altri ancora la scelta peggiore. Ma nel contesto indicato la nostra esperienza maturata in oltre dieci anni di lavoro sul campo ci ha portato a quell'indicazione.

In definitiva Regola kit è una selezione di tecnologie impiegate con successo in progetti critici di tipo gestionale e costituisce, per così dire, un prontuario di tecnologie che hanno superato la prova dei fatti nell'ambito indicato. Oltre a questo Regola kit si propone come una soluzione ingegneristica in cui tutte le citate tecnologie si trovano già configurate per funzionare correttamente assieme evitando agli utilizzatori il lungo percorso del tuning dei vari file di configurazione. Come si dice un ambiente pronto per l'uso o chiavi in mano.
L'unica parte di Regola kit che aggiunge qualcosa a l'esistente è nota come Model Pattern e colma un vuoto esistente per la selezione e la rappresentazioni di oggetti persistiti in database ricchi di dati in modo che siano facili da individuare, estrarre e riportare a video. Se in futuro qualcuno facesse meglio di noi in questo campo sicuramente il nostro Model Pattern verrebbe sostituito.

Tutto qui; Regola kit è semplicemente questo. Non sappiamo dire se sia poca cosa o tanto, comunque se può esservi di un qualche aiuto per il vostro lavoro allora usatelo liberamente (è licenziato sotto GPL3), ci farete cosa gradita!  Il manuale che avete tra le mani  fornisce una guida per lo sviluppatore con cui ci scusiamo preventivamente per le eventuali inesattezze presenti e, anzi, lo sproniamo a segnalarcele. Vorrei ringraziare Lorenzo Bragaglia, Marco Rosi, Fabio Cognigni, Alex Landini e  tutti quelli che hanno aiutato in questa impresa. E' stato davverobello lavorare con voi. In particolare voglio ricordare Lorenzo per le revisioni continue di questo documento ed  consigli preziosi.

\begin{flushright}
\textit{17 novembre 2007, Nicola Santi}
\end{flushright}





\addcontentsline{toc}{subsection}{Prefazione}

% -----------CAPITOLI
\chapter{Getting Started}

Questo capitolo è una cura per gli impazienti; seguendo le istruzioni dei paragrafi seguenti sarete in grado di installare e predisporre una prima applicazioni web con Regola.

\section{Installare Regola kit}
\index{installazione} 
Le applicazioni realizzate con Regola kit utilizzano Maven 2 per tutta la gestione del ciclo di build (compilazione, esecuzione dei test, creazione dei file war, \ldots). Quindi come prima cosa bisogna installare Maven 2 
scaricandolo dal sito \htmladdnormallink{maven.apache.org} {http://maven.apache.org} e seguendo le istruzioni.

Finita l'installazione di Maven 2 verificate che tutto sia andato a buon fine aprendo una console di terminale e lanciando il seguente comando.

\begin{bash}
nicola@casper:~# mvn -version
Maven version: 2.0.8
Java version: 1.6.0_03
OS name: "linux" version: "2.6.22-14-generic" arch: "i386" Family: "unix"
\end{bash}

Se tutto è corretto Maven 2 risponde al comando restituendo la sua versione, quella di Java ed infine alcune informazioni circa il sistema operativo in uso.

Regola kit non richiede nessuna installazione particolare (anche se è possibile scaricare un pacchetto contente documentazione e comandi di utilità): quindi finita l'installazione di Maven 2 siete pronti già per utilizzare Regola kit.


%\setlength{\fboxrule}{0.1pt} 
%\framebox[\width]{
%\begin{minipage}{0.9\textwidth}
\begin{nota}
Per maggiori informazioni su come installare Regola kit sulle vostre macchine di sviluppo si rimanda al capitolo  \vref{chap:installazione}.
\end{nota}
%\end{minipage}
%}


\section{Predisporre il database}
Per questo tutorial ipotizziamo di avere a disposizione un database di tipo MySql già installato sulla macchina dove intendiamo creare il progetto. Assicuratevi che il database stia funzionando e digitate il comando seguente:

\begin{bash}
nicola@casper:~# mysql -u root -p
\end{bash}

Vi troverete dentro la shell di amministrazione di MySql. Approfittatene per creare un nuovo database che sarà utilizzato dall'applicazione digitando il comando.

\begin{bash}
mysql> create database clienti;
\end{bash}

(Attenzione al punto e virgola in fondo al comando).
A questo punto non ci resta che creare anche l'utente utilizzato dalla nostra applicazione per accedere al database (nell'esempio porta il mio nome \emph{nicola}).

\begin{bash}
mysql> grant all on clienti.* to 'nicola'@'localhost';
\end{bash}

Bene, con il database abbiamo finito. Digitate questo ultimo comando per uscire dalla shell di amministrazione di MySql e passare al paragrafo seguente.

\begin{bash}
mysql> exit
\end{bash}

\begin{nota}
Regola kit è in grado di utilizzare diversi DBMS (ad esempio Oracle, Microsoft Sql, PostgreSQL, Hypersonic, \ldots). Per sapere come configurare la vostra applicazione per utilizzare DBMS diversi da MySql si rimanda al capitolo \vref{chap:persistenza}.
\end{nota}



\section{Creare un progetto con Regola kit}\index{progetto!creazione}

Posizionatevi nella cartella dove volete creare il vostro nuovo progetto e digitate un comando simile al seguente con l'accortezza però di modificare il parametro gruopId (nell'esempio \emph{com.acme}) con il nome del vostro package di default ed il parametro  artifactId (nell'esempio  \emph{clienti}) con il nome del nuovo progetto.

\begin{bash}
nicola@casper:~# mvn archetype:create -DarchetypeGroupId=org.regola  \ 
-DarchetypeArtifactId=regola-jsf-archetype \
-DarchetypeVersion=1.1-SNAPSHOT -DgroupId=com.acme \
-DartifactId=clienti
\end{bash}

\emph{Attenzione: il comando qui sopra è riparito su diverse righe per chiarezza tipografica, deve invece essere digitato su una sola riga.}

La prima volta che lanciate questo comando Maven 2 scarica tutte le librerie necessarie (la cosa potrebbe prendere un po' di tempo) e crea una sotto cartella col nome del progetto (nell'esempio la cartella clienti). Il progetto è questo punto è già stato creato, posizioniatevi all'interno della cartella  \emph{clienti} col comando:

\begin{bash}
nicola@casper:~# cd clienti
\end{bash}


\section{Collegarsi al database}\label{sec:db-link}\index{database!run-time config}

Prima di lanciare la nostra nuova applicazione è necessario informarla circa le coordinate del database da utilizzare, per farlo bisogna apportare un modifica al file src/test/resources/jetty/env.xml con il vostro editor di testo preferito. Dovete inserire cambiare solo il nome del database (alla riga 6) e lo username (riga 8) da utilizzare per ottenere qualcosa di simile al frammento di xml seguente:

\begin{xml}
...
<New id="jira-ds" class="org.mortbay.jetty.plus.naming.Resource">
  <Arg>jdbc/Datasource</Arg>
  <Arg>
    <New class="org.enhydra.jdbc.standard.StandardConnectionPoolDataSource">
      <Set name="Url">jdbc:mysql://localhost/clienti</Set>
      <Set name="DriverName">com.mysql.jdbc.Driver</Set>
      <Set name="User">nicola</Set>
    </New>
  </Arg>
</New>
...
\end{xml}

\begin{nota}
Per avere maggiori informazioni sulle diverse configurazioni relative alle connessioni al database si rimanda al capitolo \ref{chap:database}
\end{nota}


\section{Avviare l'applicazione}\index{applicazione!run}
Ora tutto è pronto per avviare l'applicazione, se avete lasciato la cartella principale del progetto tornateci e da lì lanciate il comando seguente (e lasciate aperta la console):

\begin{bash}
nicola@casper:~/projects/clienti# mvn jetty:run
\end{bash}

Sullo schermo si susseguiranno diverse righe per informarvi che l'applicazione è stata inizializzata e quando, infine, apparirà la dicitura \emph{Started Jetty Server} saprete che tutto è pronto.

Lasciando sempre aperta la console aprite un'istanza del vostro browser e collegatevi all'indirizzo \htmladdnormallink{localhost:8080/clienti} {http://localhost:8080/clienti} per vedere la pagina di benvenuto della vostra applicazione.

Complimenti, avete appena fatto il primo passo nel mondo delle applicazioni Regola kit!

\begin{nota}
Per visualizzare l'applicazione state utilizzando un piccolo (ma molto completo) application server di nome Jetty. Per la messa in produzione però si consiglia di utilizzare dei container diversi (ad esempio Tomcat o JBoss). Per imparare a come creare i pacchetti per questi application server si rimanda al capitolo \vref{chap:produzione} 
\end{nota}

\section{Struttura di un progetto Regola kit}\index{progetto!struttura}
La struttura della cartella di un progetto Regola kit si impronta alla struttra standard di un progetto web di Maven 2. Al primo livello troviamo:

\begin{center}
{
  \begin{tabular}{ | l | p{9cm} | }
  \hline
  pom.xml & il file di configurazione di Maven 2 \\ \hline
  src/ & la cartella dei sorgenti  \\ \hline
  target/ & contiene i file compilati ed i pacchetti per le consegne \\ \hline
  \end{tabular}
}
\end{center}

La cartella target contiene quanto i pacchetti pronti per la consegna con la classi compilate ed i descrittori. Si tratta di una cartella il cui contenuto è ricreato ogni volta si lanci il comando:

\begin{bash}
nicola@casper:~/projects/clienti# mvn package
\end{bash}

La cartella src contiene i sorgenti (html, js, css e java) dell'applicazione. Al suo interno potete trovare:

\begin{center}
{
  \begin{tabular}{ | l | p{9cm} | }
  \hline
  main/ & contiene i sorgenti dell'appplicazione \\ \hline
  test/ & contiene i sorgenti dei test \\ \hline
  site/ & reportistica generata da Maven 2  \\ \hline
  \end{tabular}
}
\end{center}

La cartella main e la cartella test contengono entrambe i sorgenti java (nella sottocartella java) e le altre risorse (nella cartella risorse). Queste ultime sono i file di configurazione, i mappaggi orm e, in generale, tutto quello che non sono sorgenti Java ma devono finire comunque nel classpath. La differenza tra la cartella main e quella test e che il contenuto di quest'ultima non finisce mai nei pacchetti destinati alla produzione ma è usato esclusivamente per l'esecuzione dei test.
Infine la cartella main contiene anche la sottocartella webapp dove si trova la webroot, ovvero le pagine web dell'applicazione ed il file web.xml.

\begin{nota}
Per una descrizione completa dei file e delle cartelle standard di un progetto Regola kit si rimanda al capitolo \vref{chap:struttura}
\end{nota}


\section{Persistenza su database}

Regola kit abbraccia una metodologia di sviluppo incentrata sull'analisi del dominio del problema (Domain Driven Development) per cui il primo passo è quello di creare le classi di modello. Spesso queste classi sono persistite sul database per cui si potrebbe iniziare scrivendo la classe e poi creando la corrispondente tabella sul database. Oppure, al contrario come faremo tra poco, creando prima la tabella del database e facendoci poi creare in automatico la classe Java (nel prossimo paragrafo \vref{sec:gsmodello}). Colleghiamoci nuovamente al database clienti.

\begin{bash}
nicola@casper:~/projects/clienti# mysql -u root -p clienti
\end{bash}

Creiamo una piccola tabella con la sola chiave primaria (id) ed un campo di descrizione (label).

\begin{bash}
mysql> create table prodotti (id int(11) not null auto_increment, label varchar(80) not null, primary key (id) );
mysql> describe prodotti;
+-------+-------------+------+-----+---------+----------------+
| Field | Type        | Null | Key | Default | Extra          |
+-------+-------------+------+-----+---------+----------------+
| id    | int(11)     | NO   | PRI | NULL    | auto_increment | 
| label | varchar(80) | NO   |     |         |                | 
+-------+-------------+------+-----+---------+----------------+
2 rows in set (0.02 sec)
\end{bash}

Inseriamo qualche dato nella tabella, ad esempio alcune descrizioni di esempio per verificare poi il funzionamento dell'applicazione.

\begin{bash}
mysql> insert into prodotti values (null, 'book');
Query OK, 1 row affected (0.05 sec)

mysql> insert into prodotti values (null, 'bottle');
Query OK, 1 row affected (0.00 sec)

mysql> insert into prodotti values (null, 'paper');
Query OK, 1 row affected (0.00 sec)

mysql> select * from prodotti;
+----+--------+
| id | label  |
+----+--------+
|  1 | book   | 
|  2 | bottle | 
|  3 | paper  | 
+----+--------+
3 rows in set (0.01 sec)
\end{bash}

Adesso il database contiene una tabella con dei dati che possiamo usare per persistere le nostre classi di modello. Usciamo dal database e torniamo all'applicazione.

\begin{bash}
mysql> exit
\end{bash}

\section{(Ri)collegarsi al database}\index{database!design-time config}
Al paragrafo \vref{sec:db-link} abbiamo configurato l'applicazione per utilizzare il nostro database in fase di esecuzione. Adesso dobbiamo fare in modo che anche in fase di sviluppo si possa accedere al database (ad esempio per usare i generatori di codice o lanciare la batteria di test). Il file da modificare è src/test/resources/designtime.properties e deve essere aggiornato in modo da contenere lo username, la password ed il nome del database. Il risultato finale deve risultare simile al seguente:

\begin{bash}
...
hibernate.dialect=org.hibernate.dialect.MySQLDialect
hibernate.connection.driver_class = com.mysql.jdbc.Driver
hibernate.connection.url = jdbc:mysql://localhost/clienti
hibernate.connection.username = nicola
hibernate.connection.password = 
...
\end{bash}


\section{Classi di modello}\label{sec:gsmodello}\index{generatori!modello}\index{modello!generazione}\index{reverse engineering}
Siete ora in grado di scrivere la classe di modello che sarà persistita sulla tabella prodotti... oppure potete farvela generare automaticamente e poi modificare convenientemente le classi prodotte. Userete gli Hibernate Tools che sono già configurati all'interno delle applicazioni prodotte con Regola kit e trovano nell'unico file src/test/resources/hibernate.reveng.xml la configurazione di tutto il processo di generazione inversa, a partire cioè dal database. Specificate il nome della tabella da cui partire (\emph{prodotti} alla riga 2), il nome della classe (\emph{Prodotto}, al singolare e con la prima lettera maiuscola nella riga 4), il package da utilizzare (\emph{com.acme.model} sempre alla riga 2).

\begin{bash}
...  
  <table-filter match-name="prodotti"  package="com.acme.model" exclude="false"/>

   <table name="prodotti" class="Prodotto" >
      <primary-key property="id" />
   </table>
...
\end{bash}

Ora avviate la generazione: posizionantevi nella cartella principale del vostro progetto ed utilizzate il plugin di Maven 2 Hibernate3 che consente di generare le classi java (il goal hbm2java), i file di mappaggio di hibernate (hbm2hbmxml) e la configurazione generale di hibernate (hbm2cfgxml). 

\begin{bash}
nicola@casper:~/projects/clienti# mvn hibernate3:hbm2java hibernate3:hbm2hbmxml hibernate3:hbm2cfgxml
\end{bash}

Il primo file generato si trova nella posizione src/main/java/com/acme/model/Prodotto.java e contiene la classe Java:


\begin{java}
public class Prodotto  implements java.io.Serializable {
    
    private Integer id;

    public Prodotto() {
    }

    public Integer getId() {
        return this.id;
    }
    
    public void setId(Integer id) {
        this.id = id;
    }
}
\end{java}

Poi è stato creato il file con i mappaggi di hibernate src/main/resources/com/acme/model/Prodotto.hbm.xml, molto semplice in questo caso:

\begin{xml}
<?xml version="1.0"?>
<!DOCTYPE hibernate-mapping PUBLIC "-//Hibernate/Hibernate Mapping DTD 3.0//EN"
"http://hibernate.sourceforge.net/hibernate-mapping-3.0.dtd">
<!-- Generated 14-apr-2008 12.23.38 by Hibernate Tools 3.2.0.CR1 -->
<hibernate-mapping>
    <class name="com.acme.model.Prodotto" table="prodotti" catalog="clienti">
        <id name="id" type="java.lang.Integer">
            <column name="id" />
            <generator class="identity" />
        </id>
    </class>
</hibernate-mapping>
\end{xml}

Ed infine è stato inserito un riferimento a quest'ultimo file di mappaggio dentro la configuraizone principale di Hibernate    src/main/resources/hibernate.cfg.xml (alla riga 13). 

\begin{xml}
<?xml version="1.0" encoding="utf-8"?>
<!DOCTYPE hibernate-configuration PUBLIC
"-//Hibernate/Hibernate Configuration DTD 3.0//EN"
"http://hibernate.sourceforge.net/hibernate-configuration-3.0.dtd">
<hibernate-configuration>
    <session-factory>
        <property name="hibernate.connection.driver_class">com.mysql.jdbc.Driver</property>
        <property name="hibernate.connection.password"></property>
        <property name="hibernate.connection.url">jdbc:mysql://localhost/clienti</property>
        <property name="hibernate.connection.username">nicola</property>
        <property name="hibernate.dialect">org.hibernate.dialect.MySQLDialect</property>
       
       <mapping resource="com/acme/model/Prodotti.hbm.xml" />
    </session-factory>
</hibernate-configuration>
\end{xml}

Attenzione: il goal hibernate3:hbm2cfgxml cancella e riscrive ogni volta questo file ed inoltre vi aggiunge delle configurazioni che a runtime non sono usate (come username, password, url e driver class). Nell'impiego di tutti i giorni di Regola kit il nostro team di sviluppo non utilizza il goal hibernate3:hbm2cfgxml e si occupa di aggiungere manualmente i mappaggi delle risorse. Naturalmente le configurazioni non usate non costituiscono problema, per cui alla fine la scelta di impiegare o meno hibernate3:hbm2cfgxml è lasciata alla vostra discrezione.

\begin{nota}
Esistono altri goal disponibili, ad esempio la generazione degli script sql per creare le tabella a partire dalla configurazione delle classi. Si rimanda, per approfondimenti, al capitolo \vref{chap:persistenza}.
\end{nota}

\section{Dal modello alla presentazione}
Adesso che avete creato la classe di modello è il momento di realizzare il codice per leggere e scrivere oggetti (della classe Prodotto) sul database, le pagine web che elenchino questi oggetti così come la pagine di dettaglio per effettuare modifiche. Questo codice può essere scritto a mano oppure potete partire facendovi generare automaticamente della classi di default che utilizzerete come modello di partenza per le vostre modifiche.

\index{generatori!master/detail} Per avviare il generatore di Regola kit utilizzate il seguente comando:

\begin{bash}
nicola@casper:~/projects/clienti# mvn exec:java -Dexec.args="-c com.acme.model.Prodotto -m"
\end{bash}

Noterete tra i parametri passati al comando il nome della classe attorno a cui costruire i vari livelli e l'opzione \emph{m} che specifica di utilizzare un'ampia catena di generatori, in particolare:

\begin{center}
{
  \begin{tabular}{ | l | p{9cm} | }
  \hline
  dao &  produce il custom dao \\ \hline
  modelPattern & produce la classe necessaria a Model Pattern   \\ \hline
  properties & aggiunge le chiavi per la localizzazione \\ \hline
  list-handler & genera il controller dietro la pagina di lista \\ \hline
  list &  genera la pagina di lista\\ \hline
  form-handler & genera il controller dietro la pagina di dettaglio\\ \hline
  form &  genera la pagina di dettaglio \\ \hline
  \end{tabular}
}
\end{center}

\begin{nota}
I generatori possono anche essere avviati individualmente. Per scoprire come e conoscere anche altri generatori forniti con Regola kit si rimanda al capitolo \vref{sec:generatori}.
\end{nota}







\chapter{Installazione}\label{chap:installazione}

\section{Maven 2}
TODO: Qui si spiega che Regola kit si fonda su Maven 2 per tutta la gestione del ciclo di build.

\section{Librerie}
TODO: Dove si elencano i moduli di cui si compone

\section{Dipendenze}
TODO: Dove si elencano le dipendenze, ricordando però che sono gestite da Maven 2

\section{Eclipse IDE}
TODO: Come lavorare ad un progetto di Regola utilizzando Eclipse 3.3 ed i plugin necessari per il funzionamento
\chapter{Struttura di un progetto}\label{chap:struttura}
Questo capitolo funziona un po' come una cartina stradale e rimanda ad altri capitoli per l'approfondimento.

\begin{center}
{
  \begin{tabular}{ | l | l | p{6cm} | }
  \hline
  / & pom.xml  & il file principale di Maven 2 \\ \hline

  src/main/resources/ & runtime.properties  & proprietà di runtime \\ 
                      & log4j.xml & la verbosità dei log \\ 
                      & hibernate.cfg.xml & la configurazione principale di Hibernate  \\ 
                      & ApplicationResources.properties & i testi tradotti nelle varie lingue  \\ 
                      & applicationContext-*.xml & le configurazioni di Spring  \\ 
                      & <stesse cartelle dei packages> & i mappaggi di Hibernate  \\ \hline

  src/test/resources/ & designtime.properties  & proprietà a design time \\ 
                      & log4j.xml & la verbosità dei log \\ 
                      &  hibernate.reveng.xml & Hibernate tools  \\ 
                      & applicationContext-*-test.xml & le configurazioni di Spring  \\ 
                      & jetty/env.xml & datasource per Jetty  \\ \hline
  \end{tabular}
}
\end{center}


\section{Lo standard Maven 2}
Fra i tanti vantaggi offerti di cui si può beneficiare utilizzando Maven 2 vi sono quelli derivanti dall'adesione ad uno standard (che i teorici dei giochi definiscono equilibrio di Nash); infatti disponendo i diversi file di un progetto in modo convenuto consente agli sviluppatori (e i sistemisiti) di orientarsi da subito su un progetto estraneo, a sistemi di supporto alla scrittura di trovare senza configurazione le risorse di cui abbisognano ed in generale consentono a soggetti diversi e lontani (nello spazio e nel tempo) di cooperare sullo stesso progetto senza conflitti.\\

La struttura esatta di progetto web è reperibile su \cite{maven}, però in estrema sintesi, la gerarchie delle cartelle prevede al primo livello la cartella dei sorgenti (src) e quella con gli artifatti prodotti (target), ad esempio le classi compilate ed i war file. La cartella src è primariamente suddivisa in due sottocartelle speculari, una contiene sorgenti di produzione (main) ed una quelli di test. Queste sottocartelle (di main e test) contengono i sorgenti java (cartella java) e le risorse (dentro resources), prevalentamente i file di configurazione. Da rilevare che la cartella main contiene anche la sottocartella webapp dove si trova la webroot, ovvero le pagine web dell'applicazione ed il file web.xml.


\section{L'iniezione delle dipendenze}
TODO: La distinzione tra i tre livelli di bean

\section{La localizzazione}
La localizzazione consiste nelle tecniche per tradurre i testi, l'aspetto e le form di input di un'applicazione nelle diverse lingue. Regola kit utilizza i Resource Bundle di Java per risolvere questo aspetto come indicato in \cite{i18n}. Nella cartella src/main/resources sono presenti diversi file  di proprietà il cui nome presenta la radice comune ApplicationResources, ad esempio:

\begin{itemize*}
  \item ApplicationResources en.properties
  \item ApplicationResources it.properties
\end{itemize*}


Queste file replicano la stessa chiave traducendola nella varie lingue. Ad esempio dentro ApplicationResources en possiamo trovare la chiave errors.required a cui è associato un testo inglese:

\begin{xml}
 errors.required=the field is required.
\end{xml}

Nel file ApplicationResources it sarà possibile fornire una traduzione per la chiave errors.required semplicemente ridefinendola.

\begin{xml}
 errors.required=campo necessario.
\end{xml}

Affinché questo meccanismo di localizzazione funzioni Regola kit provvede automaticamente a selezionare il giusto file in base alle impostazioni del browser che utilizza l'applicazione. Inoltre è necessario rammentarsi di non includere mai direttamente dei testi all'interno delle pagine web ma di richiamare le chiavi definite dentro i file ApplicationResources. Ad esempio nelle pagine jsp  bisogna inserire qualcosa di simile a:

\begin{java}
  <fmt:message key="errors.required"/>
\end{java}

Per le pagine jsf è invece possibile utilizzare il managed bean mgs.

\begin{java}
  #{msg[errors.required]}
\end{java}


\section{Le connessioni al database}
Le connessioni al database sono di due tipi a seconda dell'ambiente in cui sta girando l'applicazione. Per l'ambiente di run-time è necessario indicare il nome JNDI del datsource fornito dall'application server mentre a design-time bisogna proprio impostare tutte le caratteristiche di una connessione. In entrami i casi la configurazione riguarda l'impostazione di proprietà di configurazione dentro i file design-time.properties e run-time.properties. \\
Per maggiori dettagli ed esempi di configurazione per i diversi application server si rimanda a \vref{chap:database}.


\section{Verbosità dei log}
TODO: log4j e come interagisce con JBoss

\section{La sezione dei test}
TODO: come configurare i test e quali file utilizzano.

\section{Applicazione Servlet}
TODO: Davvero in breve la struttura
\chapter{Database}\label{chap:database}\index{datasource|see{database}}

\section{Configurazione di run-time}\index{database!run-time config}
Le applicazioni JEE generalmente lasciano la gestione delle connessioni database al container. In questo modo è possibile per i sistemisti modificare, ad esempio, la url di connessione oppure il numero di connessioni in pool senza modificare l'applicazione né dovere effettuare una riconsegna (redeploy). Ogni container configura in modo diverso però ogni datasource è caratterizzato necessariamenta da un nome JDNI, ad esempio java:comp/env/jdbc/miodatabase. Questo nome è utilizzato dall'applicazione per recuperare la connessione e, in definitiva, connettersi al database. 
Nelle applicazioni Regola kit il nome JNDI del datasource è specificato nella proprietà di runtime jee.datasrouce (nel file  src/main/resource/runtime.properties). 

\begin{bash}
jee.datasource=java:comp/env/jdbc/Datasource
\end{bash}

Con questo la configurazione di runtime della vostra applicazione è terminata, non ci sono altre variabili da impostare. Un problema tipico di configurazione riportato di diversi utenti è l'impossibilità di connettersi al datasource per avere utilizzato un indirizzo JNDI sbagliato. La causa del problema  è che il nome con cui l'application server espone la connessione non è \emph{esattamente} quello specificato nella configurazione: perché ad esempio viene preposta la stringa `java:` o a volte `java:comp/env/`.  
\\\index{database!esempi!JBoss}
Per facilitare la soluzione di questo tipo di problema concludiamo il paragrafo riportando qualche configurazione di datasource per gli application server più diffusi. Ad esempio  JBoss richiede di ricopiare nella cartella di deploy del server utilizzato un file con il nome del tipo *-ds.xml (ad esempio miadatasource-ds.xml) con un contenuto simile al seguente:

\begin{xml}
<?xml version="1.0" encoding="UTF-8"?>

<datasources>
 <local-tx-datasource>
    <jndi-name>jdbc/services</jndi-name> 
    
    <connection-url>jdbc:oracle:thin:@133.222.0.1:1522:SID</connection-url>
    <user-name>username</user-name>
    <password>*****</password>
  
    <driver-class>oracle.jdbc.driver.OracleDriver</driver-class>
    <exception-sorter-class-name>
       org.jboss.resource.adapter.jdbc.vendor.OracleExceptionSorter
    </exception-sorter-class-name>
    <min-pool-size>5</min-pool-size>
    <max-pool-size>20</max-pool-size>

 </local-tx-datasource>
</datasources>
\end{xml}

In questo caso nella configurazione dell'applicazione dovete indicare il nome indicato però preceduto da `java:` come indicato di seguito.

\begin{bash}
jee.datasource=java:jdbc/services
\end{bash}


\index{database!esempi!Jetty}Per quanto riguarda Jetty il file di configurazione env.xml contiene l'indicazione del datasource.

\begin{xml}
<?xml version="1.0"?>
<!DOCTYPE Configure PUBLIC "-//Mort Bay Consulting//DTD Configure//EN" "http://jetty.mortbay.org/configure.dtd">
<Configure class="org.mortbay.jetty.webapp.WebAppContext">

  <New id="jira-ds" class="org.mortbay.jetty.plus.naming.Resource">
     <Arg>jdbc/Datasource</Arg>
       <Arg>
         <New class="org.enhydra.jdbc.standard.StandardConnectionPoolDataSource">
           <Set name="Url">jdbc:mysql://localhost/clienti</Set>
           <Set name="DriverName">com.mysql.jdbc.Driver</Set>
           <Set name="User">nicola</Set>
         </New>
      </Arg>
  </New>

</Configure>

\end{xml}

Jetty aggiunge al nome configurato la stringa `java:comp/env/` per cui l'indirizzo JNDI da utilizzare è il seguente:

\begin{bash}
jee.datasource=java:comp/env/jdbc/Datasource
\end{bash}

TODO: Aggiungere la configurazione per Tomcat

TODO: Mi chiedo come mai il dialetto sia specificato come proprietà di designtime ma non come proprietà di runtime.



\section{Configurazione di design-time}\index{database!design-time config}
Tra le caratteristiche invidiabili del framework Spring vi è la possibilità di testare l'applicazione fuori dal container con notevoli risparmio di tempo e relativo aumento di produttività. I servizi necessari ai test sono offerti direttamente da Spring che si sostituisce al container, ad esempio, nella gestione delle transazioni e nei datasource. In quest'ultimo caso è necessario specificare tutte le proprietà di una connessione (come il driver, la url, la username e la password) come proprietà di designtime (nel file src/test/resources/designtime.properties).

\begin{bash}
...
hibernate.dialect=org.hibernate.dialect.MySQLDialect
hibernate.connection.driver_class = com.mysql.jdbc.Driver
hibernate.connection.url = jdbc:mysql://localhost/clienti
hibernate.connection.username = nicola
hibernate.connection.password = 
...
\end{bash}

Da notare che è necessario specificare anche la variabile hibernate.dialect con il tipo di \emph{dialetto} di database utilizzato. I tipi disponibili sono molti ed elencati nella documentazione di Hibernate, alcuni esempi comuni sono MySQL5Dialect, OracleDialect, PostgreSQLDialect, HSQLDialect e SQLServerDialect.




\chapter{Persistenza}\label{chap:persistenza}

\section{Hibernate}
TODO: Qui si dice che l'orm predefinito da Regola kit è Hibernate, perché è estremamente flessibile e su database preesistenti consente di gestire anche i casi più anomali. Perché non usiamo ancora JPA (immaturo).

\section{Configurazione}
TODO: I file hibernate.cfg.xml ed i mappaggi. Alcuni esempi di base per questi file.

\section{Generatori automatici}
TODO: Esempi di configurazione di hibernate.reveng.xml ed esempi (tutti) dei vari goals disponibili a riga di comando.

\section{Altri ORM}
TODO: Si parla del supporto sperimentale per gli altri orm.
\chapter{Messa in produzione}\label{chap:produzione}

\section{Produrre i pacchetti war ed ear}
TODO: Qui si spiega come produrre i pacchetti e cosa contengano al loro interno

\section{Application Server}
TODO: Perché non usare Jetty ed esempi di configurazione per Tomcat e JBoss. Si rimanda al capitolo dei database per la configurazione dei datasource.

\section{Integrazione continua}
TODO: Perché serve e come configurare i file standard dei progetti di Regola kit
\chapter{Sviluppo}
Questo capitolo apre la parte dedicata alle funzionalità offerte da Regola kit relativamente allo sviluppo. Lasciandosi alle spalle le configurazioni.

\section{Domain Driven Development}
TODO: Molto in breve si spiega come ci si incentri sul modello.

\section{Livelli}
TODO: Si presentano i vari livelli e si rimanda a capitoli successivi per i dettagli

\section{Model Pattern}
TODO: Si descrive la principale innovazione di Regola kit nell'ambito dello sviluppo (cioè in tutto quello che non riguarda sistemistica e configurazione). Qui si descrive l'idea di base del Model Pattern che sarà poi dettagliato nei paragrafi successi.

\section{Generatori}\label{sec:generatori}
TODO: Qui si elencano i generatori disponibili e cosa scrivano. Se il paragrafo diventasse troppo lungo allora lo mettiamo su un capitolo a parte.
\chapter{Dao}

\section{Scopo}
TODO: A cosa serve DAO?

\section{GenericDao}
TODO: Se ne descrivono interfacce e si presentano esempi d'uso (dei test d'unità) che si snodano tra i vari paragrafi.

\section{Creare un custom dao}
TODO: l'interfaccia, l'implementazione ed infine la configurazione di Spring. Si ricorda che esiste un generatore per questo.

\section{Ricerche con Model Pattern}\label{sec:modelpattern}
Nel paragrafo \vref{sec:modelpatternTeoria} è stato presentato in modo formale un nuovo design pattern chiamato Model Pattern; è stato spiegato come intenda risolvere il probema dell'estrazione di un sottoinsieme di oggetti e di come intenda fornire una rappresentazione sintetica di questo oggetto. La soluzione proposta, si è visto, prevede l'utilizzo di una classe che svolga il ruolo di ModelPattern, ovvero sia contemporaneamente in grado di individuare quali elementi estrarre e di rappresentarli senza però contenere al suo interno né il sottoinsieme né la logica per l'estrazione. La discussione è rimasta però a livello astratto rimandando la descrizione di come Model Pattern possa essere utilizzato concretamente all'interno di un'applicazione Regola kit; nei prossimi paragrafi vedremo quindi l'implementazione di riferimento di Model Pattern contenuta in Regola kit (precisamente nei moduli regola-core, regola-dao e sottomoduli) per scoprire come progettare un classe ModelPattern con diversi criteri di filtraggio e ordinamento e la utilizzeremo lungo i diversi livelli applicativi, dal DAO alla presentazione. Per maggiore concretezza immagineremo di doverci occupare, diciamo, della classe Prodotto riportata di seguito.

\begin{java}
class Prodotto {
    Integer id;
    String  descrizione;

    //seguono getter/setter per ogni campo
    ...
}
\end{java}

\subsection{ModelPattern}
Regola kit fornisce una classe di base org.regola.model.ModelPattern da cui derivare il nostro ModelPattern; questa classe presenta alcune facilitazioni per specificare gli ordinamenti e la paginazione ed è accettata quasi in ogni livello applicativo, dai GenericDao fino a controllori web che si occupano di disegnare la tabella con il sottoinsieme. Tenendo a mente la classe Prodotto una prima versione del nostro ModelPattern potrebbe essere la seguente:

\begin{java}
class ProdottoPattern extends org.regola.model.ModelPattern 
    implements Serializable {
    
    Integer  chiave;

    @Equals("id")
    public Integer getChiave() {
        return chiave;
    }

    ...
}
\end{java}

Soffermiamoci un attimo su alcuni elementi che trasformano la nostra classe in un ModelPattern utilizzabile con Regola kit:

\begin{description*}
  \item[nome della classe] una convenzione di Regola kit prevede che ogni ModelPatter dovrebbe chiamarsi col nome della classe di modello a cui si riferisce (nel nostro caso Prodotto) seguito dal suffisso Pattern, per cui ProdottoPattern
  \item[derivazione] la nostra classe deve derivare dalla classe org.regola.model.ModelPattern fornita con Regola kit
\item[serilizzazione] la nostra classe deve essere progettata per la serializzazione e quindi necessariamente implementare l'interfaccia di marker java.io.Serializable
\end{description*}

La classe ProdottoPattern così realizzata è predisposta per individuare particolari sottoinsiemi di instanze Prodotto che presentano particolari valori della proprietà id, in particolare tutte le istanze di Prodotto che hanno id uguale alla proprietà chiave di ProdottoPattern. Il legame tra chiave ed id è realizzato mediante l'annotazione @Equals (alla riga 6) che, posta sul getter della proprietà chiave di ProdottoPattern, unisce quest'ultima alla proprietà dell'oggetto di modello specificata dentro l'annotazione stessa (in questo caso id).
\\
Bisogna chiarire da subito che non è la classe ModelPattern ad individuare un sottoinsieme ma ogni sua istanza per cui, tornando al nostro esempio, bisogna istanziare un oggetto del tipo ProdottoPattern. 

\begin{java}
ProdottoPattern pattern = new ProdottoPattern();

pattern.setChiave(234532); //ora pattern rappresenta un sottoinsieme
\end{java}

Dopo l'assegnazione pattern (l'oggetto) e non ProdottoPattern (la classe) è in grado di individuare un sottoinsieme di oggetti Prodotto aventi la proprietà id uguale a 234532 (essendo id una chiave primaria il sottoinsieme conterrà al più un elemento). Da notare come l'oggetto pattern non contenga in sé il sottoinsieme ma solo la descrizione sintetica di questo; per ottenere il sottoinsieme bisogna rivolgersi alla classe ProdottoDao invocando il metodo find.

\begin{java}
ProdottoPattern pattern = new ProdottoPattern();

pattern.setChiave(234532); 

List<Prodotto> prodotti = prodottoDao.find(pattern);
\end{java}

Ora che abbiamo visto concretamente come creare ed utilizzare una classe ModelPattern possiamo scendere nel dettaglio ricordando che Model Pattern deve individuare con esattezza il sottoinsieme, in particolare deve occuparsi di ordinamento, paginazione e proiezione. Nei prossimi paragrafi quindi scopriremo come specificare:

\begin{enumerate*}
  \item la selezione da effettuare
  \item i criteri di ordinarmento
  \item la paginazione da effettuare 
  \item le proprietà da comprendere nella proiezione
  \item fornire rappresentazioni sintetiche del sottoinsieme indicato
\end{enumerate*}

\subsection{La selezione}
La selezione avviene impostando alcuni criteri di filtraggio sul ModelPattern. Ogni criterio è caratterizzato da:

\begin{enumerate*}
  \item la proprietà del modello a cui si applica
  \item il tipo di criterio (ad esempio ugualianza, appartenenza, ecc.) 
  \item il valore da utilizzare per il confronto
\end{enumerate*}

Riprendendo l'esempio del paragrafo precedente troviamo un unico criterio di filtraggio, di tipo ugualianza, che confronta la proprietà di modello id con il valore della proprietà chiave di ProdottoPattern.

\begin{java}
class ProdottoPattern extends org.regola.model.ModelPattern 
    implements Serializable {
    
    Integer  chiave;

    @Equals("id")
    public Integer getChiave() {
        return chiave;
    }

    ...
}
\end{java}

La proprietà di modello può trovarsi direttamente sulla classe radice, ovvero quella a cui riferisce ModelPattern (nell'esempio Prodotto) oppure su classi a questa collegate tramite associazioni del tipo uno-a-uno, molti-a-uno od uno-a-molti. In generale si navigano le associazioni utilizzando come separatore il punto tranne che per relazioni uno-a-molti (collezioni) dove si utilizza il simbolo [] postfisso. Nel dettaglio:

\begin{center}
{
  \begin{tabular}{ | l | l | l  | }
  \hline
  radice  & il nome della proprietà & id \\   
                                    & & nome \\   \hline
  uno-a-uno  & il punto    & indirizzo.via \\   
                         & & categoria.descrizione \\   \hline
  molti-a-uno  & il punto    & cliente.nome \\   
                           & & fattura.progressivo \\   \hline
  uno-a-molti  & [] postfisso ed il punto    & elmenti[].nome \\   
                                           & & elementi[].dettagli[].progressivo \\   \hline
  \end{tabular}
}
\end{center}

Il tipo del criterio è impostato scegliendo l'annotazione tra quelle presenti in Regola kit. Attualmente è possibile scegliere tra le annotazioni seguenti:

\begin{center}
{
  \begin{tabular}{ | l | l  | }
  \hline
    Equals & ugualianza \\ \hline
    NotEquals & disugualianza \\ \hline
    Like & il like \\ \hline
    GreatherThan & maggiore \\ \hline 
    LessThan & minore \\ \hline
    In & appartenenza \\ \hline
  \end{tabular}
}
\end{center}

Il valore di ogni criterio una proprietà di ModelPattern individuata annotandone il getter. Il tipo di questa proprietà deve corrispondere a quello del modello tranne per il criterio In per cui è necessario specificare un'array. L'esempio seguente fornisce una rappresentazione piuttosto completa di quando esposto:

\begin{java}
public class CustomerPattern extends ModelPattern 
   implements Serializable {
  
  private Integer id;

  @Equals("id")
  public .Integer getId()
  {
    return id;
  }
  
  private String firstName;
  
  @Like(value = "firstName", caseSensitive = true)
  public String getFirstName()
  {
    return firstName;
  }
    
  private String[] lastNames;
  
  @In("lastName")
  public String[] getLastNames() {
    return lastNames;
  }

  private Integer invoiceId;
 
  @Equals("invoices[].id")
  public Integer getInvoiceId() {
    return invoiceId;
  }

}
\end{java}

\subsection{Ordinamento}
Sugli oggetti selezionati è possibile impostare criteri di ordinamento in base alle proprietà della radice del modello o sugli oggetti ad essa associati. L'implementazione di Model Pattern presente in Regola kit utilizza per specificare gli ordinamenti (così come le proiezioni) la classe ModelProperty, una specie di descrittore di una generica proprietà. Si instazia così:

\begin{java}
   ModelProperty mp =  new ModelProperty("id","customer.column.",Order.asc);   
\end{java}

ModelProperty contiene il nome della proprietà (in base a cui ordinare), un prefisso da utilizzare per individuare in modo univoco la proprietà all'interno di tutta l'applicazione ed, infine, la direzione dell'ordinamento (acendente o discendente). 
Nell'esempio si esprime un ordinamento ascentente sulla proprietà id, individuata nell'applicazione come "customer.column.id".
\\
Per speficiare l'ordinamento basta popolare la lista sortedProperties di ModelPattern con tante istanze di ModelProperty in base all'ordinamento da realizzare. Ad esempio:

\begin{java}
   ModelProperty id =  new ModelProperty("id","customer.column.",Order.asc);
   ModelProperty street =  new ModelProperty("address.street","customer.column.",Order.desc);

   modelPattern.getSortedProperties.clear();
   modelPattern.getSortedProperties.add(id);
   modelPattern.getSortedProperties.add(street);
\end{java}

In questo caso si impone un ordinamento crescente per id e descrescente per la proprietà strret dell'oggetto associato address.


\subsection{Paginazione}
Sugli oggetti così selezionati ed ordinati è possibile effettuare un'olteriore selezione dividendolo in blocchi contigui dette pagine contenenti al più $n$ elementi. ModelPattern consente di impostare la dimensione delle pagine così come impostare la pagina da selezionare. Ad esempio:

\begin{java}
  modelPattern.setPageSize(20);
  modelPattern.setCurrentPage(0);
\end{java}

Qui si suddivide l'insieme ordinato in pagina con al più di 20 oggetti, il primo blocco comprende gli oggetti da $[0,19]$, il secondo da $[20,39]$ e così via. Inoltre si seleziona la prima pagina (la numerazione delle pagine parte da 0), ovvero gli oggetti da $[0,19]$.

\subsection{Proiezione}
Fino a qui abbiamo visto come effettuare la selezione, ordinarla e limitarla ad un certo numero di elementi. \'{E} inoltre possibile limitare non solo il numero ma anche le proprietà del singolo oggetto del sottoinisieme. Ad esempio può essere conveniente limitarsi a considerare solo la proprietà id e name piuttosto che il complesso di tutte le proprietà della radice del modello e/o degli oggetti associati. Questo genere di limitazione si chiama proiezione.
Il meccanismo per specificare le proiezione dentro Regola kit si basa sempre sugli oggetti ModelProperty, basta popolare la lista visibleProperties di ModelPattern con le proprietà che si intende proiettare.


\begin{java}
   ModelProperty id =  new ModelProperty("id","customer.column.",Order.asc);
   ModelProperty street =  new ModelProperty("address.street","customer.column.",Order.desc);

   modelPattern.getVisibleProperties.clear();
   modelPattern.getVisibleProperties.add(id);
   modelPattern.getVisibleProperties.add(street);
\end{java}

Nell'esempio si effettua una proiezione comprendente esclusivamente le proprietà id e strett dell'oggetto associato street.

\subsection{Rappresentazione}
TODO:























\chapter{Servizio}

\section{Scopo}
TODO: A cosa serve il livello di Servizio? \cite{spring}

\section{GenericManager}
TODO: Come crearlo. Si ricorda che esiste un generatore per questo.

\section{Transazioni}
TODO: come sono demarcate e come aggiungere politiche diverse per aprire e chiudere transazioni

\section{Politiche di detach}
Gli EJB, fin dal loro esordio nel 1998, hanno nascosto le classi di modello al livello di presentazione ed utilizzato in loro vece delle classi apposite, i Data Transfer Object. Questi DTO avevano il compito di raccogliere una visione appiattita del modello che però contenesse tutte e sole le informazioni necessarie al livello di presentazione. Erano oggetti di passaggio, tutti dati e nessuna logica che nascondevano la complessità del modello agli strati superiori, realizzando di fatto un disaccoppiamento forte tra questi livelli. 
Nonostante questo pregio i DTO hanno contribuito molto alla fama di pesantezza degli EJB in quanto, anche in progetti piccoli, risultava evidente che il lavoro per creare, popolare e sincronizzare i DTO con la classi di modello era di gran lunga superiore ai benefici. La domanda sorta spontanea a molti era: perché non passare ai livelli superiori direttamente le classi di modello? Molti progetti hanno dimostrato che la strada è praticabile con successo e formalizzato i vari modi per farlo.
In questo paragrafo presenteremo due pattern per utilizzare direttamente le classi di modello nel livello di presentazione conosciuti rispettivamente come pojo façade e modello  esposto.

\subsection{Pojo façade}
Questo pattern per utilizzare le classi di modello mantiene l'architettura complessiva degli EJB interponendo tra la presentazione ed il modello un apposito livello (façade) il cui compito è ancora quello di fornire servizi per la presentazione. La differenza principale con gli EJB però risiede in due aspetti:
\begin{enumerate*}
  \item  Il livello façade è costituito da semplici pojo
  \item Invece che DTO si passano oggetti di modello, opportunamente trattati
\end{enumerate*}

Del primo punto bisogna sottolineare che il livello façade, come negli EJB, è ancora responsabile di aprire e chiudere le transazioni così come della sicurezza (autenticazione ed autorizzazione). La differenza è che non è necessario un container JEE per ottenere questi servizi ma sono realizzati tramite un container invertito (come Spring) e la programmazione orientata agli aspetti (AOP). Con il vantaggio di poter effettuare tutti i test fuori dal container.
Il secondo punto invece richiede alcuni approfondimenti che discuterò di seguito.

\subsubsection{incapsulamento del modello}
Il modello, secondo la definizione più accettata di Model Driven Development (MDD), è una realtà attiva in grado di utilizzare risorse esterne, ad esempio può presentare dei metodi per salvarsi su database. Se il livello di presentazione chiamasse direttamente questi metodi verrebbe meno la funzione centralizzatrice dei pojo façade ed il sistema solleverebbe diverse eccezioni legate all'assenza di transazioni attive sul livello di presentazione.
Per ovviare a questo problema esistono alcune tecniche:
\begin{description*}
  \item[convenzione] si stabilisce che nessun sviluppatore chiami mai questi metodi e si limiti ad ignorarli. Per gruppi numerosi potrebbe essere necessario rafforzare la convenzione magari marcando i metodi da ignorare con delle annotazioni ed utilizzare un compilatore ad aspetti che riporti come errore ogni chiamata a questi metodi effettuata nel livello di presentazione.
  \item[visibilità dei metodi]  ovvero marcare i metodi come private o protected. Spesso però il modello è sparso in vari package Java per cui questa possibilità non è praticabile.
 \item[utilizzare interfacce] il livello di presentazione è scritto non in termini delle classi del modello ma in termini di sotto interfacce che ne espongano solo alcuni metodi nascondendone altri. Può essere realizzato in due modi:
    \begin{description*}
      \item[realizzare le interfacce direttamente sul modello] è una strada tecnicamente complessa perché richiede di scrivere metodi covarianti, di affrontare il problema delle collezioni immutabili ed inoltre non consente comunque di esporre quei metodi che richiedevano computazioni realizzabili solo nel livello di modello (perché ad esempio richiedono l'accesso al database)
      \item[creare degli adapter] hanno lo svantaggio principale di essere terribilmente complicati e finiscono spesso per somigliare ai DTO (in termini dei soli svantaggi).
    \end{description*}
\end{description*}

\subsubsection{oggetti staccati (detached)}
Gli elementi del modello possono effettuare implicitamente delle richieste al motore di persistenza in modo invisibile per i client. Il caso più esemplare è quello delle collezioni che gli ORM gestiscono a volte in modalità di caricamento differito (lazy o late loading). Ovvero gli elementi delle collezione sono recuperati dal database solo al momento del primo accesso ad un elemento, in modo trasparente. Il problema è che nel livello di presentazione sessioni ORM e transazioni non sono disponibili con conseguente errore a run time. Esistono in generale due tipi di soluzioni:

\begin{description*}
  \item [Eliminare i caricamenti differiti] ovvero effettuare delle configurazioni per gli ORM (dette mappaggi) che carichino subito tutte le collezioni leggendole dal database assieme all'oggetto che le contiene. Bisogna però ricordare che  questa strada non è percorribile quando la mole dei dati è notevole o lo schema del database non lo consenta; circostanze, queste, molte frequenti nella pratica.
  \item[Fare scattare i caricamenti differiti] ovvero prima di passare al livello di presentazione gli oggetti del modello fare scattare tutti i caricamenti differiti. Spesso gli ORM hanno metodi appositi per fare questo (ad esempio Hibernate ha il metodo Hibernate.initialize() ).
\end{description*}

\subsubsection{Vantaggi di Pojo Façade}
Confrontando Pojo Façade, in particolare con gli EJB Façade, emergono diversi punti di forza che riassumo brevemente:
\begin{description*}
  \item[sviluppo più facile e (quindi) veloce] per la possibilità di testare fuori dal container e l'assenza di DTO che, come visto, possono essere molto onerosi in termini di tempo.
  \item[possono eliminare la necessità di un container] visto che non è più necessario per transazioni e sicurezza potrebbe non essere necessario utilizzare un container completo JEE nel progetto.
 \item[demarcazione flessibile delle transazioni] la configurazione tramite AOP delle transazione può avere un livello di flessibilità inaudito nel mondo EJB.
\end{description*}

Confrontando invece Pojo Façade con il pattern del modello esposto (presentato nel prossimo paragrafo) si possono isolare i seguenti vantaggi:

\begin{description*}
  \item[livello di presentazione facilitato] perché non è richiesta una tecnica particolare di gestione delle transazioni sul livello di presentazione, come vedremo
 \item[vista coerente dei dati del database] tutti i dati sono esposti dai pojo façade e sono quindi raccolti all'interno di un'unica transazione. Questo può non essere vero nel pattern del modello esposto.
\end{description*}

\subsubsection{Svantaggi di Pojo Façade}
Gli svantaggi principali rispetto agli EJB façade sono:
\begin{description*}
\item[mancanza di standard] lo standard JEE non prevede Spring (per ora e purtroppo) quindi i problemi possono essere di vario genere. In verità Spring è largamente riconosciuto nella comunità di sviluppatori tanto che  moltissime librerie o sono espressamente dedicata a Spring oppure sono facilmente integrabili. Inoltre molte configurazioni standard sono riconosciute da Spring (ad esempio le annotazioni per i web service o per la persistenza). Comunque a titolo di esempio i problemi legati alla mancanza di standard possono essere:
   \begin{description*}
     \item[ottenere il pojo façade] il client potrebbe non sapere come ottenere un'istanza del façade, ad esempio un generatore che espone una classe come end point di un servizio web potrebbe non sapere come ottenere un'istanza del pojo façade.
     \item[sicurezza] la dichiarazione dei vincoli di sicurezza avviene in modo fuori standard e quindi non portabile.
    \end{description*}
\item[transazioni iniziate su client remoti] attualmente non è supportato da framework come Spring: i pojo façade non possono partecipare a transazioni avviate esternamente.
\end{description*}

Con riferimento al pattern del modello esposto i principali svantaggi sono:
  \begin{description*}
    \item[problemi con gli oggetti staccati] come visto nei paragrafi precedenti può richiedere una codifica minuziosa e difficile accertare che tutti gli oggetti siano effettivamente staccati dal motore di persistenza. Inoltre gli errori si presentano solo a runtime, cosa che rende più difficile la loro individuazione.
   \item[incapsulamento del modello] come abbiamo visto può risultare molto complesso nascondere alcuni metodi ai client.

  \end{description*}

\subsection{Modello esposto}
Un altra tecnica per evitare l'impiego dei DTO non si limita ad utilizzare le classi di modello per la presentazione ma imbocca la strada più radicale di rinunciare ad un livello di façade fornendo direttamente accesso ad un modello non più staccato (detached) ma attivo e transazionale. 
Il modello esposto è noto anche col nome di Open Session In View o anche Open Persistence Manager In View.

\subsubsection{Quando è possibile utilizzare il modello esposto}
Per poter utilizzare questo modello sono necessarie due condizioni inderogabili per il livello di presentazione che deve infatti:
\begin{itemize*}
  \item poter gestire la sessione dell'ORM
  \item accedere da locale al modello (sono escluse connessioni remote)
\end{itemize*}
 
In assenza di queste due condizioni è necessario ripiegare su soluzioni alternative come pojo façade.

\subsubsection{Come funziona}
Per consentire al modello di funzionare nel livello di presentazione è necessario occuparsi di due aspetti fondamentali:
\begin{description*}
  \item[gestire la sessione] La sessione utilizzata dall'ORM deve rimanere aperta per tutta la durata di una richiesta HTTP o addirittura per diverse richieste. Per fare questo è una soluzione consolidata ricorrere ad un filtro HTTP che si occupi prima di mantenere aperta la sessione ed infine di chiuderla anche in presenza di eccezioni. Spring presenta già dei filtri per svolgere questa funzione, tra cui OpenSessionInViewFilter. I vantaggi dell'impiego di un filtro sono:
    \begin{description*}
      \item[sicurezza] il filtro viene sempre invocato all'inizio ed alla fine della richiesta
      \item[riusabilità] lo stesso filtro può essere riusato in diverse applicazioni
      \item[ortogonalità] è possibile aggiungere e rimuovere il filtro senza modificare il codice del livello di presentazione. 
    \end{description*}      

  \item[gestire le transazioni] La gestione delle transazioni potrebbe avvenire a livello di presentazione (ovvero essere delegata al filtro della sessione) oppure al livello del modello (ovvero gestita con AOP). Nel primo caso la transazione viene aperta all'inizio della richiesta dal filtro della sessione e chiusa al termine. Nel secondo caso (AOP) è possibile aprire e chiudere le transazioni in base ad uno schema molto più flessibile, come ad esempio attorno alle classi di modello che si occupano dei servizi.
Entrambi gli approcci presentano vantaggi e svantaggi:
\end{description*}

\subsection{Conversazioni}
Nelle applicazioni web una singola operazione di business può essere completata solo attraverso diverse richieste HTTP; ad esempio mediante la compilazione da parte dell'utente di alcune pagine in sequenza e solo al completamento dell'ultima l'operazione si considera o conclusa o annullata. In termini di interazione utente queste operazioni si chiamano conversazioni (o transazioni lunghe o anche transazioni utente) e sono gestite in modo completamente diverso nel pattern dei pojo façade piuttosto che nel modello esposto.
Nei pojo façade la sessione dell'ORM inizia e si conclude all'interno della singola richiesta HTTP, per cui all'interno di una conversazione (che si compone di diverse richiesta HTTP) vengono usate sessioni diverse. Gli oggetti di modello prima di passare al livello di presentazione sono staccati (detached) dal motore di persistenza  e poi riattaccati (reattached) nella richiesta HTTP successiva. Questa modalità operativa prevede quindi transazioni sul database limitate all'interno delle richieste HTTP e dedica molta attenzione alla fase di attach e detach degli oggetti dall'ORM.
Nel pattern del modello esposto invece si utilizza la stessa sessione dell'ORM per tutta la durata della conversazione. Gli oggetti di modello passati al livello di presentazione non sono mai staccati (detached) ma rimangono sempre nel contesto di persistenza dell'ORM. Le transazioni sul database sono invece aperte e chiuse all'interno delle richieste HTTP per evitare dei lock sulle righe delle tabelle ed il conseguente degrado per tutta l'applicazione. Per assicurare la coerenza dei dati tra le varie transazioni è necessario abilitare un meccanismo per individuare eventuali variazione sui dati trattati, ad esempio una politica di lock ottimistico. 





\section{Web Services}
TODO: esempi di configurazioni già pronte dentro Regola kit
\chapter{Presentazione Web}

\section{Scopo}
TODO: Di cosa si occupa questo livello?

\section{Tecnologie}
TODO: Panoramica brevissima su JSF, Spring WebFlow e Spring MVC.

\section{Pagina di lista}
TODO: Partendo da un esempio si provvede a spiegare quali sono i file coinvolti e come devono essere configurati. Si ricorda che esiste un generatore per questo.

\section{Pagina di form}
TODO: Partendo da un esempio si provvede a spiegare quali sono i file coinvolti e come devono essere configurati. Si ricorda che esiste un generatore per questo.

\chapter{Application mashup}

\section{Servizi Web SOAP}
Un modo molto comune per consentire ad applicazioni esterne di accedere al livello di servizio delle applicazioni scritte con Regola kit è quello di esportare uno o più bean situati in quel livello tramite servizio web (web service). Per farlo è necessario che la classe sia progettata in modo da consentire la costruzione della busta SOAP o, in generale, in modo da consentire la conversione dei parametri scambiati in una qualche forma di XML; ad esempio è buona norma utilizzare tipo serializzabili, rendere transitorie le associazioni che non si intende trasferire nel servizio, gestire le ricorsioni nel grafo di oggetti, ecc.
Regola kit utilizza CXF per gestire in toto i servizi web, per cui si rimanda alla documentazione ufficiale per le tante e utili opzioni disponibili. Di seguito presenteremo un esempio piuttosto comune di basato su JAX-WS realizzato a partire da codice Java esistente nel livello di presentazione. La prima cosa è marcare l'interfaccia del servizio con l'annotazione WebService :

\begin{java}
@WebService
public interface HelloWorldManager {
    String sayHi(@WebParam(name="text") String text);
}
\end{java}


Anche la classe che realizza questa interfaccia deve essere annotata per precisare, ad esempio, il nome del servizio stesso:

\begin{java}
@WebService(endpointInterface = "demo.service.HelloWorld", 
            serviceName = "HelloWorld")
public class HelloWorldManagerImpl implements HelloWorldManager {

    public String sayHi(String text) {
        return "Hello " + text;
    }

}
\end{java}


 A questo punto rimane da configurare il bean in modo tale che costituisca un Service End Point. Nell'esempio di seguito viene modificato il file applicationContext-services.xml in modo da prevedere anche il protocollo di sicurezza WS-Security:

\begin{xml}
<jaxws:endpoint id="helloWorld"
  implementor="demo..service.impl.HelloWorldManagerImpl" address="/services/HelloWorld">
   <jaxws:features>
     <bean class="org.apache.cxf.feature.LoggingFeature" />
   </jaxws:features>
   <jaxws:inInterceptors>
   <bean class="org.apache.cxf.ws.security.wss4j.WSS4JInInterceptor">
  <constructor-arg>
    <map>
      <entry key="action" value="UsernameToken Timestamp" />
      <entry key="passwordType" value="PasswordText" />
      <entry key="passwordCallbackClass" value="ServerPasswordCallback" />
   </map>
      </constructor-arg>
    </bean>
   </jaxws:inInterceptors>
</jaxws:endpoint>
\end{xml}

Da notare  tra le features l'abilitazione del log e tra gli inIntercaptors la configurazione della classe  WSS4JInInterceptor che predispone il servizio di sicurezza WS-Security, in questo caso basato su Timestamp e su semplice password in chiaro. Conclude il tutto la classe  ServerPasswordCallback che si occupa di verificare le password in ingresso.

\begin{java}
public class ServerPasswordCallback implements CallbackHandler {

  public void handle(Callback[] callbacks) throws IOException,
     UnsupportedCallbackException {
    
  WSPasswordCallback pc = (WSPasswordCallback) callbacks[0];
  if (pc.getIdentifer().equals("joe")) {
  if (!pc.getPassword().equals("password")) {
      throw new SecurityException("wrong password");
    }
  }
  } 
}
\end{java}

CXF consente inoltre di configurare un client per il servizio in modo analogo ance se un po' prolisso:

\begin{xml}
<bean id="prenotazioni" class="demo.service.HelloWorldManager"
  factory-bean="clientFactory" factory-method="create" />
  
<bean id="clientFactory" class="org.apache.cxf.jaxws.JaxWsProxyFactoryBean">
  <property name="serviceClass" value="it.kion.service.PrenotazioniManager" />
  <property name="address" value="http://localhost/soap/services/HelloWorld" />
  <property name="outInterceptors">
   <list>
    <bean class="org.apache.cxf.ws.security.wss4j.WSS4JOutInterceptor">
     <constructor-arg>
      <map>
    <entry key="action" value="UsernameToken Timestamp" />
    <entry key="passwordType" value="PasswordText" />
    <entry key="user" value="joe" />
        <entry key="passwordCallbackClass" value="ClientPasswordCallback" />
  </map>
     </constructor-arg>
    </bean>
    <bean class="org.apache.cxf.interceptor.LoggingOutInterceptor" />
   </list>
  </property>
  <property name="inInterceptors">
   <list>
    <bean class="org.apache.cxf.interceptor.LoggingInInterceptor" />
   </list>
  </property>
</bean>
\end{xml}


Qui è possibile vedere la configurazione per la sicurezza (speculare a quella lato server) , la configurazione del log e la classe  ClientPasswordCallback responsabile di presentare al server le password.

\begin{java}
public class ClientPasswordCallback implements CallbackHandler {

  public void handle(Callback[] callbacks) throws IOException,
      UnsupportedCallbackException {
      WSPasswordCallback pc = (WSPasswordCallback) callbacks[0];
  // set the password for our message.
  pc.setPassword("password");
   }
}
\end{java}


CXF è una libreria davvero completa che consente di realizzare servizi web in modalità diversa (ad esempio partendo dal WSDL piuttosto che dal codice Java, oppure utilizzando SOAP o servizi di tipo REST, ecc.). Si invita quindi a consultare la documentazione per avere un quadro generale delle funzioni disponibili.

\begin{nota}
Per utilizzare CXF all'intenro di JBOSS è necessario impostare il parametro di avvio della virtual machine javax.xml.soap.MessageFactory al valore com.sun.xml.messaging.saaj.soap.ver1\_1.SOAPMessageFactory1\_1Impl.
\end{nota}

\section{Servizi Web REST}
Un servizio di tipo REST può essere realizzato sempre annotando una classe di tipo manager (nel livello quindi di servizio) nel modo seguente:

\begin{java}
@Path("/nome-servizio/")
@ProduceMime("text/xml")
public class RestManager {

  @POST @Path("/foo/{id}/{nome}")
  public Dto postFoo(@PathParam("id") String id, @PathParam("nome") String nome, Dto parametro) {

    return new Dto(id + ":" + nome + ":" + parametro.prova);
  }

  @PUT @Path("/foo/{id}/{nome}")
  public Dto putFoo(@PathParam("id") String id, @PathParam("nome") String nome, Dto parametro) {

    return new Dto(id + ":" + nome + ":" + parametro.prova);
  }

  @GET @Path("/foo/{id}/{nome}")
  public String getFoo(@PathParam("id") int id, @PathParam("nome") String nome) {

    return "GET di:" + id + ":" + nome;
  }

  @DELETE @Path("/foo/{id}/{nome}")
  public String deleteFoo(@PathParam("id") int id, @PathParam("nome") String nome) {

    return "DELETE di:" + id + ":" + nome;
  }
  
}
\end{java}


Le cose importanti da sottolineare sono:
\begin{enumerate*}
\item l'annotazione @Path che specifica la url alla quale risponde il servizio ed le singole operazioni, nell'esempio /nome-servizio/foo. 
\item I parametri di tipo primitivi (stringhe, diciamo) sono passati all'operazione aggiungendoli alla url. Sempre nell'annotazione @Path è possibile indicare i nomi dei parametri ( @Path("/foo/{id}/{nome}") )che poi saranno effettivamente passati al metodo Java con l'annotazione @PathParam (ad esempio @PathParam("id") ).
\item il metodo http al quale l'operazione deve rispondere ad esempio @POST, @GET , @PUT e @DELETE
\end{enumerate*}

La registrazione del servizio in Spring richiede semplicemente qualcosa di simile ad questo:

\begin{xml}
  <jaxrs:server id="testService" address="/">
    <jaxrs:serviceBeans>
        <ref bean="testServiceImpl" />
      </jaxrs:serviceBeans>
  </jaxrs:server>

  <bean id="testServiceImpl" class="it.kion.service.RestManager" />
\end{xml}

Per invocare i servizi di tipo REST Regola kit mette a disposizione RestClient,  una piccola classe che dovrebbe rendere agevole l'invocazione di servizi. Ad esempio:
 per un invocare un servizio col metodo http GET si può procedere come segue:

\begin{java}
 RestClient client = new RestClient(); 
 String result = client.get(url, param1, param2, ..., paramN);
 
\end{java}

I parametri sono utilizzati per costruire la url che diventa qualcosa di simile ad ( url/param1/param2/.../paramN).
 E\' anche possibile passare una singola entità nel body della richiesta HTTP, tipicamente una frammento di xml che rappresenta un'istanza di un oggetto (creato con la classe di utilità di Regola kit  JAXBMarshaller):

\begin{java}
 Dto dto = new Dto();
 String dtoXml = toXml("org.regola.ws", "dto", dto);
 String result = client.post(url, dtoXml, 1, "salve!");
 
\end{java}
 
Anche il valore di ritorno dei servizi spesso sono dei documenti xml che rappresentano un oggetto; per ottenere l'oggetto di partenza si può ricorrere sempre alla classe 
 JAXBMarshaller:

\begin{java}
 String result = client.put(url, dtoXml, 1, "salve!");
 Dto dto = fromXml("org.regola.ws", result);
 
\end{java} 

Nota bene: prima di effettuare le conversioni oggetto/xml è  necessario utilizzare il compilatore di JAXB per annotare le classi coinvolte e produrre un oggetto del tipo ObjectFactory (si veda la documentazione di JAXB). Ad esempio, partendo da uno schema xml che descrive il documento relativo ad una certa classe, diciamo Dto, bisogna invocare il compilatore di JAXB come segue:

\begin{bash}
 xjc schema1.xsd -p it.il.tuo.package -d src/main/java  
\end{bash}
 


A questo punto tutte le classi coinvolte saranno create nel package it.il.tuo.package unitamente alla classe ObjectFactory.

Infine se per qualche bizzarra circostanza si disponesse solo della classe e non dello schema relativo, sarà possibile usare il metodo di utilità JAXBMarshaller.generateSchema()  per ottenere uno schema di base.


\section{Portlet}
Le pagine di delle applicazioni Regola kit possono essere fruite normalmente attraverso il browser (come abbiamo descritto nei capitoli precedenti) e, nel contempo, esportate come Portlet; questo doppio canale di fruizione offre il vantaggio di poter esportare la vostra applicazione all'interno di portale, di testare la vostra Portlet semplicemente dal browser e, infine, di poter utilizzare lo stesso stack operativo di Regola kit  (dao, manager, Model Pattern, ecc.) per realizzare Portlet.

Per trasformare una pagina web in una Portlet è necessario adottare alcune accortezza all'interno del  modello facelet (il file con estensione .xhtml) e specificare alcune configurazioni. Per quando riguardo il primo punto bisogna limitarsi ad inserire il tag ice:portlet dopo il tag f:view e prima di tutta la gerarchia di componenti. Ecco un semplice esempio di pagina abilitata ad essere un modello di Portlet:

\begin{xml}
<!DOCTYPE html PUBLIC "-//W3C//DTD XHTML 1.0 Transitional//EN" "http://www.w3.org/TR/xhtml1/DTD/xhtml1-transitional.dtd">
<html xmlns="http://www.w3.org/1999/xhtml"
  xmlns:ui="http://java.sun.com/jsf/facelets"
  xmlns:h="http://java.sun.com/jsf/html"
  xmlns:f="http://java.sun.com/jsf/core"
  xmlns:ice="http://www.icesoft.com/icefaces/component">
<body>
  <f:view>
     <ice:portlet>
        Portlet di esempio realizzata con Regola kit
     </ice:portlet>
  </f:view>
</body>
</html>
\end{xml}

Per quanto riguarda le configurazioni bisogna intervenire prima di tutto sul file /WEB-INF/portlet.xml per elencare le Portlet esportate dall'applicazione e per ciascuna indicare il nome (nell'esempio MyPortlet) e la pagina web da utilizzare come modello (ad esempio /myportlet.html). Per le altre configurazioni si rimanda alla documentazione ufficiale relativa alle Portlet. Ecco un esempio di configurazione:

\begin{xml}
<?xml version="1.0"?>
<portlet-app xmlns="http://java.sun.com/xml/ns/portlet/portlet-app_1_0.xsd" version="1.0" xmlns:xsi="http://www.w3.org/2001/XMLSchema-instance" xsi:schemaLocation="http://java.sun.com/xml/ns/portlet/portlet-app_1_0.xsd http://java.sun.com/xml/ns/portlet/portlet-app_1_0.xsd">
    <portlet>
        <portlet-name>MyPortlet</portlet-name>
        <display-name>Regola kit Portlet</display-name>
        <portlet-class>com.icesoft.faces.webapp.http.portlet.MainPortlet</portlet-class>
        <init-param>
            <name>com.icesoft.faces.VIEW</name>
            <value>/myportlet.html</value>
        </init-param>
        <supports>
            <mime-type>text/html</mime-type>
            <portlet-mode>view</portlet-mode>
        </supports>
        <portlet-info>
            <title>Regola kit Portlet Example</title>
            <short-title>My Portlet</short-title>
            <keywords>regola-kit icefaces portlet</keywords>
        </portlet-info>
    </portlet>
</portlet-app>
  
\end{xml}


A questo punto la vostra applicazione è in grado di essere consegnata dentro un portale aderente alle specifiche Portlet 1.0, ad esempio LifeRay o JeetSpeed alla cui documentazione rimandiamo per i dettagli del deploy. Di seguito però illustreremo come importare la nostra applicazione all'interno di un container Pluto che consiste semplicemente nell'aggiungere all'interno del /WEB-INF/web.xml una servelt del tipo PortletServlet associata ad ogni Portlet esposta (nell'esempio MyPortlet)  e mappata all'indirizzo  /PlutoInvoker/NomePortlet. Ecco il frammento da aggiungere al file web.xml:

\begin{xml}
<servlet>
        <servlet-name>MyPortlet</servlet-name>
        <servlet-class>org.apache.pluto.core.PortletServlet</servlet-class>
        <init-param>
            <param-name>portlet-name</param-name>
            <param-value>MyPortlet</param-value>
        </init-param>
        <load-on-startup>1</load-on-startup>
  </servlet>
  
  <servlet-mapping>
      <servlet-name>MyPortlet</servlet-name>
      <url-pattern>/PlutoInvoker/MyPortlet</url-pattern>
  </servlet-mapping>
  
\end{xml}

Infine bisogna consegnare la nostra applicazione nella stessa istanza del container (ad esempio di Tomcat 5.5) in cui sta girando il container Pluto. Ricordiamo che le applicazioni Regola kit sono configurate di default per essere un container Pluto (oltre che un fornitore di Portlet)  e quindi possono visualizzare Portlet di altre applicazioni (realizzate o meno con Regola kit). Vediamo come.

\section{Mashup}
Un'applicazione realizza un mashup quando le proprie pagine raccolgono all'interno frammenti di altre applicazioni. Regola kit realizza il mashup di Portlet e fornisce strumenti per agevolare la realizzazioni di Portlet come descritto nel paragrafo precedente. Le applicazioni Regola kit sono di default dei contenitori di Portlet basati su Pluto, quindi non è necessario realizzare nessuna configurazione per abilitare questa funzionalità tranne ricordarsi di abilitare, a livello di container, la funzionalità di accedere al altri contesti. In Tomcat 5.5 questo si realizza specificando nel descrittore di contesto qualcosa di simile a questo:

\begin{xml}
<Context crossContext="true" />
\end{xml}

Ricordiamo che il contesto si può configurare dentro la cartella conf di Tomcat, oppure dentro conf/Catalina/localhost o addirittura dentro la web root nel file META-INF/context.xml. 

I Portlet ad importare all'interno della vostra applicazione devono essere consegnati nello stesso application server dove gira il container. Attualmente è possibile utilizzare esclusivamente delle pagine jsp per effettuare il mashup dei vari portlet. Ecco un esempio di pagina in cui viene inclusa una Portlet chiamata MyPortlet, fornita da un'applicazione che si trova nel  contesto homes:

\begin{xml}
<%@ taglib uri="http://java.sun.com/jstl/core" prefix="c" %>
<%@ taglib uri="http://portals.apache.org/pluto" prefix="pluto" %>
<html>
<head>
    <title>Prova</title>
    <style type="text/css" title="currentStyle" media="screen">
        @import "<c:out value="${pageContext.request.contextPath}"/>/pluto.css";
        @import "<c:out value="${pageContext.request.contextPath}"/>/portlet-spec-1.0.css";
    </style>
    <script type="text/javascript" src="<c:out value="${pageContext.request.contextPath}"/>/pluto.js"></script>
</head>
    <body>
         <h1>Questa pagina include la portlet con nome MyPortlet</h1>
         <pluto:portlet portletId="/homes.MyPortlet">
            <pluto:render/>
          </pluto:portlet>
   </body>
</html>
\end{xml}




\chapter{Sicurezza}

\section{Presentazione}

Il sistema di sicurezza è costruito attorno ad una catena di oggetti chiamati filtri (attenzione, non si tratta di filtri Http) che vengono eseguiti in cascata e cooperano per assolvere i diversi compiti nei quali la funzione della sicurezza si concretizza. Esistono molti filtri standard ma è comunque possibile aggiungerne di personalizzati ad esempio per aggiungere un sistema di autenticazione proprietario o non ancora implementato. L'elenco che segue indica i filtri standard nell'ordine con cui sono eseguiti e per ciascuno il nome della classe che lo implementa ed  un alias per riferirsi ad esso:

\begin{tabular}{ | l | l | }
  \hline
  CHANNEL\_FILTER & ChannelProcessingFilter \\ \hline
  CONCURRENT\_SESSION\_FILTER  & ConcurrentSessionFilter \\ \hline
  SESSION\_CONTEXT\_INTEGRATION\_FILTER  & HttpSessionContextIntegrationFilter  \\ \hline
  LOGOUT\_FILTER & LogoutFilter  \\ \hline
  X509\_FILTER X509 & PreAuthenticatedProcessigFilter \\ \hline
  PRE\_AUTH\_FILTER & Subclass of AstractPreAuthenticatedProcessingFilter\\ \hline
  CAS\_PROCESSING\_FILTER & CasProcessingFilter \\ \hline
  AUTHENTICATION\_PROCESSING\_FILTER & AuthenticationProcessingFilter \\ \hline
  BASIC\_PROCESSING\_FILTER & BasicProcessingFilter \\ \hline
  SERVLET\_API\_SUPPORT\_FILTER & classname \\ \hline
  REMEMBER\_ME\_FILTER & RememberMeProcessingFilter \\ \hline
  ANONYMOUS\_FILTER & AnonymousProcessingFilter \\ \hline
  EXCEPTION\_TRANSLATION\_FILTER & ExceptionTranslationFilter \\ \hline 
  NTLM\_FILTER & NtlmProcessingFilter \\ \hline
  FILTER\_SECURITY\_INTERCEPTOR & FilterSecurityInterceptor \\ \hline
  SWITCH\_USER\_FILTER & SwitchUserProcessingFilter \\ \hline
\end{tabular}

Prima di calarci nel dettaglio dei filtri più importanti conviene concentrarci sulla configurazione più importante del sistema, ovvero quella che specifica quali di questi filtri utilizzare per una certa url e che si concretizza nella definizione del bean filterChainProxy:  


\begin{xml}
<bean id="filterChainProxy"
     class="org.acegisecurity.util.FilterChainProxy">
  <property name="filterInvocationDefinitionSource">
     <value>
    CONVERT_URL_TO_LOWERCASE_BEFORE_COMPARISON
    PATTERN_TYPE_APACHE_ANT 
    /images/**=#NONE#
    /scripts/**=#NONE# 
    /styles/**=#NONE#
    /**=httpSessionContextIntegrationFilter,logoutFilter,authenticationProcessingFilter,securityContextHolderAwareRequestFilter,anonymousProcessingFilter,exceptionTranslationFilter,filterInvocationInterceptor
    </value>
  </property>
</bean>
\end{xml}

L'aspetto è piuttosto intimidatorio sulle prime ma in effetti si limita a specificare le catene di filtri da adottare per le url indicate, nell'esempio dispone di non utilizzare nessun filtro (\#NONE\#) per le url /images/**, /scripts/**, e /style/**. Per tutte le altre url (/**) invece si utilizza la catena dei sette filtri specificati. Si può notare che le url sono indicate utilizzando la notazione di ANT essendo stato passato il parametro   PATTERN\_TYPE\_APACHE\_ANT, alternativamente si potevano utilizzare le espressioni regolari. 
Come si diceva ogni filtro assolve un compito specifico e deve essere configurato per modificarne comportamento; ad esempio l'ultimo dei sette filtri specificati nella catena,  filterInvocationInterceptor, si occupa di stabilire quali utenti (o quali gruppi di utenti) possano accedere alle pagine protette. Si può configurare come segue:

\begin{xml}
<bean id="filterInvocationInterceptor"
  class="org.acegisecurity.intercept.web.FilterSecurityInterceptor">
  <property name="authenticationManager"
 
                  ref="authenticationManager" />
  <property name="accessDecisionManager"
        ref="accessDecisionManager" />
  <property name="objectDefinitionSource">
    <value>
    PATTERN_TYPE_APACHE_ANT
    /services/*=ROLE_ANONYMOUS,admin,user
    /login.*=ROLE_ANONYMOUS,admin,user
    /**/*.html*=admin,user
    </value>
  </property>
</bean>
\end{xml}

Iniziamo dalla proprietà dal nome oscuro  objectDefinitionSource che specifica una serie di url (sempre nel formato ANT) e per ciascuno quale utente o ruolo ha il permesso di accesso; ad esempio le url del tipo /services/* possono essere accedute da tutti (utenti anonimi cioè quelli con il ruolo ROLE\_ANONYMOUS, l'utente admin e l'utente user). Stessa cosa per la pagina di login mentre per tutte le altre pagine l'accesso è consentito solo agli utenti user ed admin. Da notare come queste regole siano applicate in cascata, così come quelle specificate nella configurazione della catena dei filtri. 
Come spesso accade la configurazione di un bean ne richiede altri, nel caso di   filterInvocationInterceptor si utilizzano anche i bean  authenticationManager e  accessDecisionManager. Trattiamo prima quest'ultimo che ha il compito di stabilire  se un utente possa o meno accedere; di solito si utilizza una configurazione standard per in base al ruolo si può accedere o meno. La configurazione è la seguente.

\begin{xml}
<bean id="accessDecisionManager"
  class="org.acegisecurity.vote.AffirmativeBased">
  <property name="allowIfAllAbstainDecisions" value="false" />
  <property name="decisionVoters">
    <list>
      <bean class="org.acegisecurity.vote.RoleVoter">
        <property name="rolePrefix" value="" />
      </bean>
    </list>
  </property>
</bean>
\end{xml}

L'altro bean utilizzato per configurare  filterInvocationInterceptor è  authenticationManager, si tratta di un bean importantissimo che è utilizzato anche da altri filtri (ad esempio  authenticationProcessingFilter, il terzo filtro della catena di sette configurata sopra). Il bean  authenticationManager si occupa infatti di fornire i componenti che effettuano l'autenticazione dell' utente corrente (ad esempio ma non necessariamente utilizzando username e password) e fornire l'oggetto che rappresenta quell'utente ed i ruoli ricoperti all'interno del sistema di sicurezza. Ora, dal momento che i metodi per autenticare un utente potrebbero essere diversi, ad esempio prima su una database poi in caso di fallimento su un altro poi in caso di fallimento tramite LDAP e così via  authenticationManager contiene la lista di questi servizi di autenticazione chiamati authentication provider:

\begin{xml}
<bean id="authenticationManager"
  class="org.acegisecurity.providers.ProviderManager">
  <property name="providers">
    <list>
       <ref local="daoAuthenticationProvider" />
       <ref local="anonymousAuthenticationProvider" />
       <ref local="rememberMeAuthenticationProvider" />
    </list>
  </property>
</bean>
\end{xml}

Come si può immaginare esistono già pronti per l'uso diversi authentication provider come l' anonymousAuthenticationProvider che provvede le credenziali per l'utente anonimo, la cui configurazione si limita a dare un nome (anonymous)  a questo particolare utente.

\begin{xml}
<bean id="anonymousAuthenticationProvider"
    class="org.acegisecurity.providers.anonymous.AnonymousAuthenticationProvider">
  <property name="key" value="anonymous" />
</bean>
\end{xml}

Invece l' authentication provider  rememberMeAuthenticationProvider si occupa di autenticare un utente in base ad un cookie precedentemente salvato sul browser, la cui configurazione (riutilizzando l'uthenticationManager ) non è riportata di seguito per evitare di perdere il filo del discorso.
Ci occupiamo invece del  daoAuthenticationProvider  perché ricopre un ruolo di rilievo nel sistema di sicurezza. Infatti la classe  DaoAuthenticationProvider è in grado di utilizzare tutti i meccanismi di autenticazioni basati su username e password forniti col sistema semplicemente delegando l'individuazione dell'utente e del ruolo ad un bean chiamato userDetailsService.

\begin{xml}
<bean id="daoAuthenticationProvider"
  class="org.springframework.security.providers.dao.DaoAuthenticationProvider">
  <property name="userDetailsService" ref="inMemoryDaoImpl"/>
  <property name="saltSource" ref="saltSource"/>
  <property name="passwordEncoder" ref="passwordEncoder"/>
</bean>    
\end{xml}
  
Nell'esempio lo userDetailsService utitlizzato è inMemoryDaoImpl che si limita a leggere utenti e password direttamente dal file di configurazione.

\section{Configurazione semplificata}

Nel paragrafo precedente abbiamo specificato la catena dei filtri da applicare per ogni url e poi indicato per ogni url quale utente o gruppo pottesse accedere. Infine abbiamo fornito un semplice sistema di autenticazione basato su un file di configurazione di utenti, gruppi e password. In molti hanno pensato che, per quanto flessibile, il sistema di configurazione sopra descritto risulti davvero complicato e finalmente nella versione 2.0 (oltre al cambio di nome da Acegi a Spring Security) è stato introdotto un sistema di configurazione basato su configurazioni di default ed implementato utilizzando un namespace specifico. Il risulto è che tutta la configurazione descritta precedentemente può essere realizzato come segue:

\begin{xml}
<http auto-config='true'>

<intercept-url pattern="/**" access="#NONE#" />
<intercept-url pattern="/**" access="#NONE#" />
<intercept-url pattern="/**" access="#NONE#" />
<intercept-url pattern="/services/*" access="ROLE_ANONYMOUS,admin,user" />
<intercept-url pattern="/**/*.html*" access="admin,user" />
</http>

<authentication-provider>
   <user-service>
    <user name="jimi" password="jimispassword" authorities="ROLE_USER, ROLE_ADMIN" />
    <user name="bob" password="bobspassword" authorities="ROLE_USER" />
   </user-service>
</authentication-provider>
\end{xml}

La cosa importante da segnalare è che la nuova configurazione non sostituisce la precedente ma la realizza in modo automatico; ad esempio il tag <http> ed <intercept-url> definiscono implicitamente  filterInvocationInterceptor specificando una catena di filtri di default  mentre <authentication-provider> crea un daoAuthenticationProvider e <user-service> un  userDetailsService del tipo InMemoryDaoImpl. Questi authentication provider sono poi associati ad un bean authenticationManager proprio come quello specificato precedentemente.

Si rimanda poi alla documentazione ufficiale per esempi in cui si mescolano le due modalità di configurazione in modo da ridurre al minimo il codice necessario a mettere in sicurezza la vostra applicazione.


% -----------BIBLIOGRAFIA

%\begin{thebibliography}{9}
%\bibitem{lamport94}
%  Leslie Lamport,
%  \emph{\LaTeX: A Document Preparation System}.
%  Addison Wesley, Massachusetts,
%  2nd Edition,
%  1994.
%\end{thebibliography}

\bibliographystyle{alpha}
\bibliography{main}



% -----------INDICE
\printindex

\end{document}
