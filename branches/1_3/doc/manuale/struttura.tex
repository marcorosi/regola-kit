\chapter{Struttura di un progetto}\label{chap:struttura}
Questo capitolo funziona un po' come una cartina stradale e rimanda ad altri capitoli per l'approfondimento.

\begin{center}
{
  \begin{tabular}{ | l | l | p{6cm} | }
  \hline
  / & pom.xml  & il file principale di Maven 2 \\ \hline

  src/main/resources/ & runtime.properties  & proprietà di runtime \\ 
                      & log4j.xml & la verbosità dei log \\ 
                      & hibernate.cfg.xml & la configurazione principale di Hibernate  \\ 
                      & ApplicationResources.properties & i testi tradotti nelle varie lingue  \\ 
                      & applicationContext-*.xml & le configurazioni di Spring  \\ 
                      & <stesse cartelle dei packages> & i mappaggi di Hibernate  \\ \hline

  src/test/resources/ & designtime.properties  & proprietà a design time \\ 
                      & log4j.xml & la verbosità dei log \\ 
                      &  hibernate.reveng.xml & Hibernate tools  \\ 
                      & applicationContext-*-test.xml & le configurazioni di Spring  \\ 
                      & jetty/env.xml & datasource per Jetty  \\ \hline
  \end{tabular}
}
\end{center}


\section{Lo standard Maven 2}
Fra i tanti vantaggi offerti di cui si può beneficiare utilizzando Maven 2 vi sono quelli derivanti dall'adesione ad uno standard (che i teorici dei giochi definiscono equilibrio di Nash); infatti disponendo i diversi file di un progetto in modo convenuto consente agli sviluppatori (e i sistemisiti) di orientarsi da subito su un progetto estraneo, a sistemi di supporto alla scrittura di trovare senza configurazione le risorse di cui abbisognano ed in generale consentono a soggetti diversi e lontani (nello spazio e nel tempo) di cooperare sullo stesso progetto senza conflitti.\\

La struttura esatta di progetto web è reperibile su \cite{maven}, però in estrema sintesi, la gerarchie delle cartelle prevede al primo livello la cartella dei sorgenti (src) e quella con gli artifatti prodotti (target), ad esempio le classi compilate ed i war file. La cartella src è primariamente suddivisa in due sottocartelle speculari, una contiene sorgenti di produzione (main) ed una quelli di test. Queste sottocartelle (di main e test) contengono i sorgenti java (cartella java) e le risorse (dentro resources), prevalentamente i file di configurazione. Da rilevare che la cartella main contiene anche la sottocartella webapp dove si trova la webroot, ovvero le pagine web dell'applicazione ed il file web.xml.


\section{L'iniezione delle dipendenze}
TODO: La distinzione tra i tre livelli di bean

\section{La localizzazione}
La localizzazione consiste nelle tecniche per tradurre i testi, l'aspetto e le form di input di un'applicazione nelle diverse lingue. Regola kit utilizza i Resource Bundle di Java per risolvere questo aspetto come indicato in \cite{i18n}. Nella cartella src/main/resources sono presenti diversi file  di proprietà il cui nome presenta la radice comune ApplicationResources, ad esempio:

\begin{itemize*}
  \item ApplicationResources en.properties
  \item ApplicationResources it.properties
\end{itemize*}


Queste file replicano la stessa chiave traducendola nella varie lingue. Ad esempio dentro ApplicationResources en possiamo trovare la chiave errors.required a cui è associato un testo inglese:

\begin{xml}
 errors.required=the field is required.
\end{xml}

Nel file ApplicationResources it sarà possibile fornire una traduzione per la chiave errors.required semplicemente ridefinendola.

\begin{xml}
 errors.required=campo necessario.
\end{xml}

Affinché questo meccanismo di localizzazione funzioni Regola kit provvede automaticamente a selezionare il giusto file in base alle impostazioni del browser che utilizza l'applicazione. Inoltre è necessario rammentarsi di non includere mai direttamente dei testi all'interno delle pagine web ma di richiamare le chiavi definite dentro i file ApplicationResources. Ad esempio nelle pagine jsp  bisogna inserire qualcosa di simile a:

\begin{java}
  <fmt:message key="errors.required"/>
\end{java}

Per le pagine jsf è invece possibile utilizzare il managed bean mgs.

\begin{java}
  #{msg[errors.required]}
\end{java}


\section{Le connessioni al database}
Le connessioni al database sono di due tipi a seconda dell'ambiente in cui sta girando l'applicazione. Per l'ambiente di run-time è necessario indicare il nome JNDI del datsource fornito dall'application server mentre a design-time bisogna proprio impostare tutte le caratteristiche di una connessione. In entrami i casi la configurazione riguarda l'impostazione di proprietà di configurazione dentro i file design-time.properties e run-time.properties. \\
Per maggiori dettagli ed esempi di configurazione per i diversi application server si rimanda a \vref{chap:database}.


\section{Verbosità dei log}
TODO: log4j e come interagisce con JBoss

\section{La sezione dei test}
TODO: come configurare i test e quali file utilizzano.

\section{Applicazione Servlet}
TODO: Davvero in breve la struttura