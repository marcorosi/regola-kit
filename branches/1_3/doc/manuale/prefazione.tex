\chapter*{Prefazione}

Il panorama delle tecnologie Java si presenta allo sviluppatore rigoglioso e selvaggio come doveva apparire agli occhi dei primi europei la foresta amazzonica; centinaia di librerie si contendono la stessa funzionalità ed assolvono il loro compito in modo spesso diverso e non di rado incompatibile. Una menzione particolare merita lo sviluppo di applicazioni web per cui esistono paradigmi e librerie a bizzeffe in una sfilata allegra e caotica di sigle (Servlet, JSP, Struts, JSF, Ajax, Tapestry e molte molte altre) che lasciano spesso interdetto il neofita, che di fronte a tanta abbondanza si ritrova spesso a scegliere a caso lasciandosi guidare dall'ultimo articolo sfogliato. Purtroppo la statistica non perdona e spesso la scelta si rivela al di sotto delle aspettative ed è qui, sopra i tavoli di una qualche riunione, che qualcuno cede allo scoramento e pronuncia la fatidica frase: <<Riscriviamolo noi>>. Ahi quanti si sono imbarcati nell'impresa funesta di realizzare il proprio framework in casa e quanti ancora ne stanno pagando un caro prezzo in termini di produttività, innovazione e stabilità.

Regola kit è un tentativo di risposta alla domanda “Quale tecnologie conviene utilizzare per un'applicazione web con un'alta interazione utente e mole di dati ampissima?”. Ad esempio per un gestionale web, il dominio che ha fatto da culla per il progetto. Conviene subito fare chiarezza su un punto: Regola kit non è un framework. Il suo contenuto è essenzialmente esperienza. Il filosofo Hume suggeriva un metodo curioso per selezionare i libri in un biblioteca; scartare tutti quelli che contengono speculazioni astratte, trattati di logica e quant'altro per limitarsi a considerare solo quei testi che riportano esperienze reali degli autori. Questo criterio di selezione vale più che mai in un mondo in cui librerie e progetti (open source o commerciali) sorgono e tramontano nello spazio di un mattino e fornisce la chiave per selezionare ciò che in potenza appare perfetto da quello che realmente è in grado di consentire lo sviluppo con un livello di produttività e gratificazione decenti. Ad esempio per la sola persistenza esistono decine di framework ma quello scelto per Regola kit (Hibernate) è l'unico che ha superato la prova su database piuttosto estesi, il giusto compromesso tra flessibilità ed automatismo. Naturalmente per altri contesti Hibernate potrebbe non essere la soluzione migliore e per altri ancora la scelta peggiore. Ma nel contesto indicato la nostra esperienza maturata in oltre dieci anni di lavoro sul campo ci ha portato a quell'indicazione.

In definitiva Regola kit è una selezione di tecnologie impiegate con successo in progetti critici di tipo gestionale e costituisce, per così dire, un prontuario di tecnologie che hanno superato la prova dei fatti nell'ambito indicato. Oltre a questo Regola kit si propone come una soluzione ingegneristica in cui tutte le citate tecnologie si trovano già configurate per funzionare correttamente assieme evitando agli utilizzatori il lungo percorso del tuning dei vari file di configurazione. Come si dice un ambiente pronto per l'uso o chiavi in mano.
L'unica parte di Regola kit che aggiunge qualcosa a l'esistente è nota come Model Pattern e colma un vuoto esistente per la selezione e la rappresentazioni di oggetti persistiti in database ricchi di dati in modo che siano facili da individuare, estrarre e riportare a video. Se in futuro qualcuno facesse meglio di noi in questo campo sicuramente il nostro Model Pattern verrebbe sostituito.

Tutto qui; Regola kit è semplicemente questo. Non sappiamo dire se sia poca cosa o tanto, comunque se può esservi di un qualche aiuto per il vostro lavoro allora usatelo liberamente (è licenziato sotto GPL3), ci farete cosa gradita!  Il manuale che avete tra le mani  fornisce una guida per lo sviluppatore con cui ci scusiamo preventivamente per le eventuali inesattezze presenti e, anzi, lo sproniamo a segnalarcele. Vorrei ringraziare Lorenzo Bragaglia, Marco Rosi, Fabio Cognigni, Alex Landini e  tutti quelli che hanno aiutato in questa impresa. E' stato davverobello lavorare con voi. In particolare voglio ricordare Lorenzo per le revisioni continue di questo documento ed  consigli preziosi.

\begin{flushright}
\textit{17 novembre 2007, Nicola Santi}
\end{flushright}



