\chapter{Installazione}\label{chap:installazione}

\section{Maven 2}
TODO: Qui si spiega che Regola kit si fonda su Maven 2 per tutta la gestione del ciclo di build.

\section{Librerie}
TODO: Dove si elencano i moduli di cui si compone

\section{Dipendenze}
TODO: Dove si elencano le dipendenze, ricordando però che sono gestite da Maven 2

\section{Eclipse IDE} \index{IDE!Eclipse}
Per lavorare su un progetto Regola kit con Eclipse basta lanciare il comando seguente che provvede a creare tutti i file necessari a quell'ambiente:

\begin{bash}
nicola@casper:~/projects/yourProject# mvn eclipse:eclipse
\end{bash}

A questo punto basta importare il progetto nel workspace con la voce di menu   File | Import : magari senza settare l'opzione di copia e quindi lavorando sul progetto nella sua cartella originale.

Per installare i plugin necessari al funzionamento dell'assistente alla scrittura del codice (vedi il paragrafo a pagina  \vref{sec:generatori} )  basta semplicemente ricopiare il file ... dentro la cartella eclipse/plugin e riavviare Eclipse.

